\documentclass{beamer}
\usetheme{Warsaw}
\usecolortheme{beaver}
\usepackage[]{algorithm2e}
\usepackage{amsmath}

\begin{document}
\title[Scalable EM Alignment] % (optional, only for long titles)
{Scalable Alignment of Electron Microscope Image Sections}
\author[Zou, Ni] {Roger Zou \and Julia Ni}
\date{\today}

\frame{\titlepage}

\begin{frame}
\frametitle{Table of Contents}
\begin{itemize}
\item Overview
\item Pairwise Alignment
\item Global Stack Alignment
\item Other Attempted Methods
\item References
\item Acknowledgements 
\end{itemize}
\end{frame}

\begin{frame}
\frametitle{Overview}
\begin{itemize}
\item \textbf{Goal:} Alignment of 3D EM sections, integration with CAJAL-3D API \\
\item \textbf{Method:} Cross-correlation to align pairwise images, then globally align entire image cube \\
\item \textbf{Complexity:} \textit{O(nlogn)}
\end{itemize}
\end{frame}

\begin{frame}
\frametitle{Pairwise Alignment}
\begin{itemize}
\item Generate transformation parameters via cross-correlation.
\item Identify peak using Support Vector Machine classifier.
\item Correct rotation as needed through error minimization.
\item Refine transformations using image data outside pairwise images.
\end{itemize}
\end{frame}

\begin{frame}
\frametitle{Pairwise Alignment}
\framesubtitle{Using Correlation}
For each pair of images: 
\begin{enumerate}
\item Apply median filtering, histogram equalization, and hamming window. 
\item Take Discrete Fourier Transform, apply high-pass filter, and resample in log-polar coordinates. 
\item Find best $\rho$, $\theta$ by correlation and max picking. 
\item Rotate image, then correlate to find the best translation parameters.
\item Use SVM to identify peak in cross-correlation of image pair. 
\item Iterate through image stack. 
\end{enumerate}
\end{frame}

\begin{frame}
\frametitle{Pairwise Alignment}
\framesubtitle{Peak Identification: SVM} 
\begin{enumerate}
\item Train SVM classifier with selected peak features. 
\item Partition cross-correlation of pairwise images. 
\item Find maximum intensity of each partition. 
\item Classify as peak/non-peak using SVM! 
\end{enumerate}
\end{frame}

\begin{frame}
\frametitle{Pairwise Alignment}
\framesubtitle{Peak Identification: Other Attempted Methods} 
\begin{itemize}
\item Choose maximum values.
\item Correlate pairwise cross-correlation with normal distribution. 
\end{itemize}
\end{frame}

\begin{frame}
\frametitle{Pairwise Alignment}
\framesubtitle{Fine-Tuning Transformation Parameters} 
\begin{itemize}
\item Error metrics (PSNR, MSE, PixDif)
\item Adjust transformation parameters by error minimization.
\end{itemize}
\end{frame}

\begin{frame}
\frametitle{Pairwise Alignment}
\begin{itemize}
\item Solve for x_1, x_2, x_3, x_4:
	\begin{align*}
	 \text{minimize: } z &= e_{1,2,3}x_1 +  e_{2,3,4}x_2 + e_{2,3}x_3 + e_ix_4\\
	 \text{constraints: }&x_1 + x_2 + x_3 + x_4 = 1\\
	 &x_1, x_2, x_3, x_4 \ge 0
	\end{align*}
\item Plug into T_{2,3}^{*} = [T_{1,3}^{-1}*T_{1,2}]x_1 + [T_{2,4}*T_{3,4}^{-1}]x_2 + [T_{2,3}]x_3 + [I]x_4.
\end{itemize}
\end{frame}

\begin{frame}
\frametitle{Global Stack Alignment}
\begin{itemize}
\item Set previous global transformations of previous image to current image. 
\item Find new rotation angle.
\item Find new translation parameters.
\item Positive/negative translations? 
\item Iterate through image cube. 
\end{itemize}
\end{frame}

\begin{frame}
\frametitle{Other Attempted Methods}
\begin{itemize}
\item RANSAC
\item SURF feature matching
\item Superpixels and Earth Mover's Distance 
\end{itemize}
\end{frame}

\begin{frame}
\frametitle{References}
\begin{thebibliography}{9}
\bibitem{lamport94}
	B. Srinivasa Reddy and B. N. Chatterji,
	\emph{An FFT-Based Technique for Translation, Rotation, and Scale-Invariant Image Registration}.
	IIEEE Transactions on Image Processing Vol. 5, No. 8, 1996
\bibitem{rubner00}
	Yossi Rubner, Carlo Tomasi, Leonidas J. Guibas,
	\emph{The Earth Mover's Distance as a Metric for Image Retrieval}.
	International Journal of Computer Vision 40(2), 99-121, 2000
\bibitem{wang13}
	Peng Wang, Gang Zeng, Rui Gan, Jingdong Wang, Hongbin Zha,
	\emph{Structure-Sensitive Superpixels via Geodesic Distance}.
	International Journal of Computer Vision, 103:1-21, 2013
\end{thebibliography} 
\end{frame}

\begin{frame}
\frametitle{Acknowledgements}
Many thanks to:
\begin{itemize}
\item Dr. Paul Bendich 
\item Dr. Robert Calderbank
\item Will Gray 
\item Dr. John Harer
\item Kunal Lillaney
\item Kathy Peterson 
\item Dr. Guillermo Sapiro
\item Ashleigh Thomas 
\item Chris Tralie
\item Dr. Joshua Vogelstein
\item Duke Math RTG Group 
\end{itemize}
\end{frame}

\end{document}
